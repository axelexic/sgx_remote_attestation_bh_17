\documentclass[11pt]{article}
\usepackage{cite}
\usepackage{fancyhdr}
\usepackage{url}
\usepackage{tikz}
\usepackage{hyperref}
\topmargin=-5mm
\evensidemargin=0cm
\oddsidemargin=0cm
\textwidth=16cm
\textheight=22cm
\addtolength{\headheight}{1.6pt}
\hypersetup{pdfstartview=}
\newcommand{\lastupdate}{\today}
\newcommand{\ie}{\textit{i.e.}}
% \lhead{\sc On the insufficiency of Intel SGX Remote Attestation}
% \rhead{\sc \lastupdate}

\title{\bf A Framework for analyzing Intel SGX Enclaves}
\author{\textsc{Yogesh Prem Swami}}

\date{\lastupdate}

\begin{document}
\pagenumbering{arabic}

\maketitle

\begin{abstract}
  Intel SGX enclaves  provide hardware
  enforced confidentiality and integrity guarantees for running pure
  computations (\ie, OS-level side-effect-free code) in the cloud
  environment. In addition, SGX remote attestation enables
  enclaves to prove that a claimed enclave is indeed running inside a
  genuine SGX hardware and not some (adversary controlled) SGX
  simulator.

  Since cryptographic protocols do not compose well
  \cite{ucframework}, the SGX remote attestation is only a necessary
  pre-condition for securely instantiating an enclave. In practice,
  one needs to analyze all the different interacting enclaves as part of a
  single protocol to make sure that no sub-computation of the protocol
  can be simulated outside of the enclave. In this paper, we present a
  practical framework for analyzing enclaves. We use Intel provided
  EPID\cite{epid} Provisioning Enclave (which is  only partially
  documented in \cite{sgxattest}) as an example within this framework
  and report our findings.
\end{abstract}

\section{Introduction}
  Intel SGX enclaves\cite{sgxinnov, sgxinnov2} provide hardware
  enforced confidentiality and  integrity guarantees for running pure
  computation (\textit{i.e.}, OS-level side-effect-free code) in the
  cloud environment. By limiting the application's Trusted Computing
  Base (TCB) to the CPU and CPU-Cache, SGX provides unprecidented
  confidentiality and integrity guarantees against malicious OS
  kernels and supervisor software. Since a remote user cannot be sure
  if an enclave is indeed running on a real-hardware instead of a
  software simulator (such as QEMU), SGX provides a remote attestation
  mechanism to prove to third parties that an enclave has indeed been
  instantiated on a real hardware.

  Given such strong guarantees, a popular design methodology for
  creating secure cloud applications is to:

  \begin{itemize}
    \item Compose multiple independent well-known protocols inside the
      enclave to create a larger protocol, and

    \item Depend on remote attestation to ensure that different parts
      of the protocol can only
  \end{itemize}

  Good examples of this methodology are \cite{Haven, Graphene, Scone},
  where the authors try to run unmodified (or minimally modified)
  applications inside the enclave, and depend upon remote-attestation
  to make sure that the internal state of the application will not be
  compromise.

  It's well known that
\bibliographystyle{alpha}
\bibliography{sgx_biblio}

\end{document}
